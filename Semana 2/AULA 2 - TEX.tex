%%%%%%%%%%%%%%%%%%%%%%%%%%%%%%%%%%%%%%%%%%%%%%%%%%%%%%
%%%%%%%%%%%%%%%%%%%%%%%%%%%%%%%%%%%%%%%%%%%%%%%%%%%%%%
%%% 
%%% MAF 172 - MATEMÁTICA COMPUTACIONAL
%%% AULA 2 - PROGRAMA ANALÍTICO
%%% 
%%%%%%%%%%%%%%%%%%%%%%%%%%%%%%%%%%%%%%%%%%%%%%%%%%%%%%
%%%%%%%%%%%%%%%%%%%%%%%%%%%%%%%%%%%%%%%%%%%%%%%%%%%%%%

% Aqui começa o preâmbulo do doc

\documentclass{beamer} 
\usepackage[brazil]{babel} % Pacote para reconhecer pt-br
\usepackage[utf8]{inputenc} % Pacote para reconhecer acentuação
\usepackage{amsmath} % Pacote para notação math
\usepackage{dsfont} % Pacote para símbolos math
\usepackage{graphicx} % Pacote para figuras
\usetheme{Warsaw} % Recurso que define o layout
% Faça uma busca no google por "theme latex beamer"
%\usecolortheme{beaver}
% Faça uma busca no google por "beamer color theme"
\usepackage{amsthm} % Pacote para destaques
\newtheorem{teoc}{Importante:} % Comando gerando um destaque
\usepackage{natbib} % Pacote para o uso de citações
\bibliographystyle{abbrv} % Temos que determinar um estilo
% Faça uma busca no google por "bibliographystyle latex overleaf"

%%%%%%%%%%%%%%%%%%%%%%%%%%%%%%%%%%%%%%%%%%%%%%%%%%%%%%

\title{MAF 172 - MATEMÁTICA COMPUTACIONAL}
\author{Prof. Dr. Gérson R. Santos}
\date{UFV - CAF}

% Aqui termina o preâmbulo do doc
%%%%%%%%%%%%%%%%%%%%%%%%%%%%%%%%%%%%%%%%%%%%%%%%%%%%%%
% Aqui começa o doc visual

\begin{document}

%%%%%%%%%%%%%%%%%%%%%%%%%%%%%%%%%%%%%%%%%%%%%%%%%%%%%%

\begin{frame}{Aula 2 - Programa Analítico}
    \maketitle
\end{frame}

%%%%%%%%%%%%%%%%%%%%%%%%%%%%%%%%%%%%%%%%%%%%%%%%%%%%%%

\begin{frame}{Sumário}
    \tableofcontents
\end{frame}

%%%%%%%%%%%%%%%%%%%%%%%%%%%%%%%%%%%%%%%%%%%%%%%%%%%%%%

\section{Introdução}

\begin{frame}{Motivação}
    
	\begin{teoc}
		\begin{itemize}
			\item "Eu quero seguir carreira acadêmica..."
			\item "Eu quero ser professor(a)..."
			\item "Eu quero empreender..."
			\item "Eu quero atuar no mercado de trabalho..."
			\item A flexibilidade na construção da grade curricular de MatComp possibilita oportunidades bem interessantes
			\item Conhecer o mercado, prever resultados, estimar riscos, otimizar processos, estruturar algoritmos...
			\item Aliar Computação, Estatística e Matemática com segurança!!!
		\end{itemize}
	\end{teoc}
\end{frame}

%%%%%%%%%%%%%%%%%%%%%%%%%%%%%%%%%%%%%%%%%%%%%%%%%%%%%%

\section{Programa}

\begin{frame}{Ementa}
    
	\begin{teoc}
        Editores de texto. Planilhas eletrônicas. Recursos computacionais para o ensino de Geometria e Aritmética. Pesquisa científica na web. Softwares de computação simbólica. Produção de material compartilhado
        (\cite{wanner2003introduccao}, \cite{wickham2016r} e \cite{bragancca2016introduccao}). 
	\end{teoc}
	
\end{frame}

%%%%%%%%%%%%%%%%%%%%%%%%%%%%%%%%%%%%%%%%%%%%%%%%%%%%%%

\begin{frame}{Conteúdo}
    
	\begin{teoc}
        \begin{itemize}
			\item Introdução ao LaTeX via Overleaf;
			\item Introdução ao Programa R via Rstudio;
			\item Introdução ao R-markdown;
			\item Trabalhando com o Geogebra;
			\item A utilização do Programa R na Pesquisa Operacional.
		\end{itemize}
	\end{teoc}
	
\end{frame}

%%%%%%%%%%%%%%%%%%%%%%%%%%%%%%%%%%%%%%%%%%%%%%%%%%%%%%

\section{Avaliação}

\begin{frame}{Avaliação}
    
	\begin{teoc}
        \begin{itemize}
			\item Avaliação 1 - LaTeX \textbf{(20 pontos)}
			\item Avaliação 2 - Programa R (20 pontos)
			\item Avaliação 3 - R-markdown (20 pontos)
			\item Avaliação 4 - Geogebra (20 pontos)
			\item Avaliação 5 - Pesquisa Operacional (20 pontos)
		\end{itemize}
	\end{teoc}
	
\end{frame}

%%%%%%%%%%%%%%%%%%%%%%%%%%%%%%%%%%%%%%%%%%%%%%%%%%%%%%

\section{ATENÇÃO}

\begin{frame}{Para Próxima Aula:}
    
	\begin{teoc}
        \begin{itemize}
			\item Abrir uma conta no Overleaf;
			\item Assistir a videoaula da Aula 2;
			\item Executar os mesmos procedimentos da Aula 2;
			\item Instalar o Programa R;
			\item Instalar o Programa RStudio.
		\end{itemize}
	\end{teoc}
	
\end{frame}

%%%%%%%%%%%%%%%%%%%%%%%%%%%%%%%%%%%%%%%%%%%%%%%%%%%%%%

\begin{frame}{Referencial Básico}
    \bibliography{ref}
\end{frame}

%%%%%%%%%%%%%%%%%%%%%%%%%%%%%%%%%%%%%%%%%%%%%%%%%%%%%%

\end{document}